
\usepackage{expl3}
\usepackage{xparse}

\ExplSyntaxOn

    \NewDocumentCommand \e {m} { % e-Funktion
        \symup{e}^{#1}
    }

    \NewDocumentCommand \dif {m} { 
        \symup{d}{#1}
    }

    \NewDocumentCommand \err {m} { % zur Darstellung von Fehlern
        \symup{\Delta}{#1}
    }

    \NewDocumentCommand \inv {m} { % Inverse
        \frac{1}{#1}
    }

    \NewDocumentCommand \Dif {m} { % für Ableitungen
        \frac{\symup{d}}{\dif{#1}}
    }

    \NewDocumentCommand \spi {} {
        \symup{\pi}
    }

    \NewDocumentCommand \xprod {m m}{ % Kreuzprodukt
        \symbf{#1} \times \symbf{#2}
    }

    \NewDocumentCommand \Betrag {m}{ % Betragsfunktion
        \bigl|{#1}\bigr|
    }

    \NewDocumentCommand \prozent {m}{ % Zahl mit Prozentzeichen
        \SI{#1}{\percent}
    }

    \NewDocumentCommand \sinv {m} { % schräge Inverse (wie bei sfrac)
        \sfrac{1}{#1}
    }

    \NewDocumentCommand \partDif {m m}{ % für partielle Ableitungen
        \frac{\partial^#2}{\partial #1^#2}
    }

    \NewDocumentCommand \Difk {m m} { % für höhere Ableitungen
        \frac{\symup{d}^{#2}}{\symup{d} {#1}^{#2}}
    }

    \NewDocumentCommand \Norm{m}{ % Norm
        \Betrag{\Betrag{#1}}
    }

    \NewDocumentCommand \Erw{m}{
        \bigl< #1 \bigr>
    }


\ExplSyntaxOff